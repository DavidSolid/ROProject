\documentclass[11pt,a4paper]{article}
%\usepackage{mathtools}
\usepackage{amsmath}
\usepackage{amssymb}
\title{
	Relazione di progetto didattico \\
	Ricerca Operativa \\
	A.A. 2018/2019
}
\author{Enrico Cancelli 1143080}
\begin{document}
\maketitle
\newpage
\tableofcontents
\newpage
\section{Abstract}
Lo scopo di questo progetto didattico è modellare un problema di programmazione lineare riguardante una linea di produzione di condensatori, tradurre tale modello in linguaggio \textit{AMPL} e risolverlo. In seguito verrà fatta un'analisi sui dati ottenuti, discutendo di eventuali valori spuri ottenuti.
\section{Descrizione del problema}
Un'azienda produce condensatori di due tipologie, elettrolitici e ceramici, in 3 differenti formati (o gradi di capacità): $500 \mu F$, $1000 \mu F$, $1500 \mu F$.
Ciascun tipo di condensatore viene venduto ad un certo prezzo in euro (generalmente all'aumentare della capacità corrispondono guadagni maggiori) così come riportato nella seguente tabella: \\
\begin{center}
\begin{tabular}{|c|c|c|c|}
 \hline
 Tipologia/Formato & $500 \mu F$ & $1000 \mu F$ & $1500 \mu F$ \\
 \hline
 Elettrolitico  &    30.0     &     50.0     &     70.0     \\
 \hline
 Ceramico       &    20.0     &     40.0     &     60.0     \\
 \hline
\end{tabular}
\end{center}
Ciascun condensatore, a seconda della tipologia e formato, può essere prodotto utilizzando un certo numero di unità di uno o più materiali tra i seguenti:
\begin{itemize}
	\item{Alluminio(Al)}
	\item{Ceramica}
	\item{Rame(Cu)}
\end{itemize}
Il numero di unità di materiale necessario alla produzione di una tipologia di condensatore è riportato nella seguente tabella: \\
\begin{center}
\begin{tabular}{|c|c|c|c|c|c|c|c|c|c|}
\hline
 Tipologia/Formato & \multicolumn{3}{|c|}{$500 \mu F$} & \multicolumn{3}{|c|}{$1000 \mu F$} & \multicolumn{3}{|c|}{$1500 \mu F$} \\
 \hline
                   & Al & Cer. & Cu & Al & Cer. & Cu & Al & Cer. & Cu \\
 \hline
 Elettrolitico     & 5  &    0     &  5 & 7  &     0    & 7  &  10  &     0    & 10  \\
 \hline
 Ceramico          & 2  &    2     &  2 & 4  &    4     & 4  & 5  &      5    & 5  \\
 \hline
\end{tabular}
\end{center}
\subsection{Altri Vincoli}
Il magazzino dell'azienda può contenere 2517 unità di Alluminio, 951 unità di Ceramica e 2439 unità di Rame utilizzabili. L'azienda si rifornisce di materiali tramite tre aziende fornitrici. Ciascuna azienda produce solo certi tipi di materiali e li vende ad un prezzo specifico (in euro). La seguente tabella riporta tali prezzi. \\
\begin{center}
\begin{tabular}{|c|c|c|c|}
\hline
           & Al & Cer. & Cu \\
\hline
Azienda 1  & 3.0  &  4.0   & 5.0  \\
\hline
Azienda 2  & 2.0  &  5.0   & 3.0  \\
\hline
Azienda 3  & 4.0  &  3.0   & 4.0  \\
\hline
\end{tabular}
\end{center}
 L'azienda 3 impone al cliente un contratto d'acquisto esclusivo per l'alluminio e in cambio offre un premio in denaro che ammonta a 5000 euro se la quantità di unità complessive supera 220 (Precisazione: questo premio è disponibile solo se si aderisce a tale contratto acquistando alluminio solo dall'azienda 3). L'azienda 1 invece è disposta a vendere soltanto partite in quantità superiori a 100. \\
 In totale è necessario produrre almeno 50 unità di condensatori a $500 \mu F$, 100 unità a $1000 \mu F$ e 50 unità a $1500 \mu F$. Inoltre esistono politiche aziendali che impongono che i condensatori elettrolitici costituiscano almeno il 30\% delle unità prodotte e che venga venduta una sola tipologia di condensatori a $1000 \mu F$.
 \subsection{Funzione Obiettivo}
 L'obiettivo dell'azienda produttrice di condensatori è massimizzare il guadagno (ovvero la differenza tra ricavo e costi). 
\section{Modellazione del problema}
\subsection{Principali variabili e definizione generale della funzione obiettivo}
Inizialmente si stabiliscono le variabili che rappresentano il numero di condensatori prodotti per ogni coppia tipologia-classe : \\
\begin{gather*}
C_{i,j}=\# \text{ condensatori del tipo i e di categoria j prodotti} \\
C_{i,j} \in \mathbb{Z}_+
\end{gather*}
Altre variabili utili a definire la funzione obiettivo sono quelle che rappresentano la quantità di materiali acquistati dai rispettivi produttori: \\
\begin{gather*}
M_{m,a}=\# \text{ unità di materiale m acquistato dall'azienda a} \\
M_{m,a} \in \mathbb{Z}_+
\end{gather*}
Inoltre definiamo delle costanti $V_{i,j} \in \mathbb{R}_+$ che rappresentano il prezzo di vendita per unità di prodotto e $A_{m,a} \in \mathbb{R}_+$ che rappresentano il prezzo di acquisto per unità di materiale.
La funzione obiettivo può quindi essere definita così: \\
\begin{equation}
\text{max } \underbrace{\sum_{i,j}^{} V_{i,j}*C_{i,j}}_{ricavo} - \underbrace{\sum_{m,a} A_{m,a}*M_{m,a}}_{costi}
\end{equation}
\subsection{Vincoli prodotto/materiale}
Si definisce l'insieme di costanti $E_{m,i,j} \in \mathbb{Z}_+$ che rappresentano il numero di unità di materiale m utilizzate durante la produzione di un condensatore di tipologia i e classe j.
I seguenti vincoli legano la quantità di ciascun materiale acquistato al numero di condensatori prodotti: \\
\begin{gather*}
\text{s.t. } \sum_{i,j} E_{m,i,j}*C_{i,j} \leq \sum_{a} M_{m,a} \\
\forall m
\end{gather*}
Nota: i vincoli non sono di uguaglianza. Pertanto è consentito all'azienda di effettuare un approvvigionamento superiore a quanto sia realmente necessario per la produzione (ciò potrebbe essere utile in caso di ottenimento di bonus o sconti sull'intera partita)
\subsection{Vincoli di approvvigionamento e produzione}
Innanzitutto stabiliamo i vincoli di capienza per i magazzini (le capienze massime fanno riferimento a quelle precedentemente enunciate nella descrizione del problema): \\
\begin{gather*}
\text{s.t. } \sum_{a} M_{"Alluminio",a} \leq 2517\\
\text{s.t. } \sum_{a} M_{"Ceramica",a} \leq 951\\
\text{s.t. } \sum_{a} M_{"Rame",a} \leq 2439
\end{gather*}
Stabiliamo inoltre la produzione minima per ogni classe di condensatore... \\
\begin{gather*}
\text{s.t. } \sum_{i} C_{i,"500 \mu F"} \geq 50 \\
\text{s.t. } \sum_{i} C_{i,"1000 \mu F"} \geq 100 \\
\text{s.t. } \sum_{i} C_{i,"1500 \mu F"} \geq 50 \\
\end{gather*}
..., la quota di mercato per i condensatori elettrolitici... \\
\begin{equation}
\text{s.t. } \sum_{j} C_{"Elettrolitico",j} \geq \frac{30}{100}*\sum_{i,j} C_{i,j}
\end{equation}
... e la dimensione della partita minima per l'azienda 1
\begin{equation}
\text{s.t. } \sum_{m} M_{m,"Azienda 1"} \geq 100
\end{equation}
\subsection{Vincoli con variabili logiche}
Può essere prodotta una sola tipologia di condensatori a $1000 \mu F$. Definiamo quindi una variabile booleana per ogni tipologia di condensatore: \\
\begin{gather*}
T_{i}=\left \{
\begin{tabular}{cc}
0 & $C_{i,"1000 \mu F"} = 0$ \\
1 & $C_{i,"1000 \mu F"} >0$
\end{tabular}
\right. \\
T_{i} \in \{ 0,1\}
\end{gather*}
Attiviamo tali variabili (Big è una costante molto grande): \\
\begin{gather*}
\text{s.t. } C_{i,"1000 \mu F"} \leq Big*T_{i} \\
\forall i
\end{gather*}
Ora possiamo mettere in relazione tra loro le variabili: \\
\begin{equation}
\text{s.t. } \sum_{i} T_{i} =1
\end{equation}
Infine per modellare il contratto di esclusività imposto dall'azienda 3, è necessario introdurre nuove variabili decisionali:
\begin{gather*}
U_{a}=\left \{
\begin{tabular}{cc}
0 & $M_{"Alluminio",a} = 0$ \\
1 & $M_{"Alluminio",a} >0$
\end{tabular}
\right. \\
U_{a} \in \{ 0,1\}
\end{gather*}
Attiviamo queste variabili:
\begin{gather*}
\text{s.t. } M_{"Alluminio",a} \leq Big*U_{a} \\
\forall a
\end{gather*}
Dal punto di vista logico imporre l'esclusività per un'azienda equivale a soddisfare la seguente espressione logica: \\
\begin{equation}
\underbrace{(U_{0} \lor ... U_{"Azienda 3"-1})}_{U_{ex}} \oplus U_{"Azienda 3"}
\end{equation}
Esprimiamo $U_{ex}$ tramite un vincolo: \\
\begin{equation}
\text{s.t. } \sum_{a-\{"Azienda 3"\}} U_{a} \leq \underbrace{card(a-\{"Azienda 3"\})}_{\text{oppure \#a - 1}}*U_{ex}
\end{equation}
Ora che abbiamo attivato $U_{ex}$, esprimiamo il vincolo di esclusività con $U_{"Azienda 3"}$: \\
\begin{equation}
U_{ex} + U_{"Azienda 3"} = 1
\end{equation}
Ora che abbiamo modellato la condizione di esclusività, è necessario modellare la condizione di superamento della soglia minima per la ricezione del premio in denaro. Chiamiamo $P$ una variabile booleana così definita:
\begin{gather*}
P=\left \{
\begin{tabular}{cc}
0 & $\sum_{m} M_{m,"Azienda 3"} <220$ \\
1 & $\sum_{m} M_{m,"Azienda 3"} \geq 220$
\end{tabular}
\right. \\
P \in \{ 0,1\} \\
\end{gather*}
Attiviamo $P$:
\begin{equation}
\sum_{m} M_{m,"Azienda 3"} -220\leq Big*P
\end{equation}
Creiamo un'ultima variabile decisionale $S=U_{"Azienda 3"} \land P$: \\
\begin{equation}
U_{"Azienda 3"} + P \geq 2*S
\end{equation}
Nota: in caso $U_{"Azienda 3"}$ e $P$ siano entrambi uguali a 1, S può assumere il valore spurio 0. Tuttavia siccome se $S=1$ viene garantito uno sconto sull'acquisto delle merci da Azienda 3, il valore spurio non verrà mai considerato perchè condurrà ad una soluzione sub-ottima. \\
Essendo $S$ la variabile decisionale che stabilisce se il premio sia applicabile o meno, modifichiamo la funzione obiettivo in questo modo: \\
\begin{equation}
\text{max } \underbrace{\sum_{i,j}^{} V_{i,j}*C_{i,j}}_{ricavo} - \underbrace{\sum_{m,a} A_{m,a}*M_{m,a}}_{costi} + \underbrace{Bonus*S}_{premio}
\end{equation}
\subsection{Modello generale conclusivo}
In conclusione il modello, generalizzando opportunamente alcune costanti, si presenta così: \\
\subsubsection{Insiemi(o Set)}
$I=\{"Elettrolitico","Ceramico"\}$:\text{ insieme di tipologie di condensatore} \\
$J=\{"500 \mu F","1000 \mu F","1500 \mu F"\}$:\text{ insieme di classi di capacità} \\
$M=\{"Alluminio","Ceramica","Rame"\}$:\text{ insieme dei tipi di materiali} \\
$A=\{"Azienda 1","Azienda 2","Azienda 3"\}$:\text{ insieme di aziende fornitrici}

\subsubsection{Costanti(o Parametri)}
$V_{i,j}:i \in I, j \in J =\text{ prezzo per unità di prodotto di tipo i e classe j}$ \\
$(V_{i,j} \in \mathbb{R}_+)$ \\
$A_{m,a}:m \in M, a \in A =\text{ prezzo per unità di materiale m presso l'azienda a}$ \\
$(A_{m,a} \in \mathbb{R}_+)$ \\
$E_{m,i,j}:m \in M, i \in I, j \in J =$ numero di unità di m utilizzate per la produzione di un condensatore di tipo i e classe j\\
$(E_{m,i,j} \in \mathbb{Z}_+)$ \\
$X_{m}: m \in M =$ capienza massima del magazzino per il materiale m \\
$(X_{m} \in \mathbb{Z}_+)$ \\
$M_{j}: j \in J =$ produzione minima per condensatori di classe j \\
$(M_{j} \in \mathbb{Z}_+)$ \\
$Per =$ quota di mercato minima per i condensatori elettrolitici (in percentuale) \\
$(Per \in \{0,100\} )$ \\
$K =$ partita minima acquistabile da Azienda 1 \\
$(K \in \mathbb{Z}_+)$ \\
$L =$ grandezza minima per la partita da Azienda 3 per ottenere il premio \\
$(L \in \mathbb{Z}_+)$ \\
$Bonus =$ valore del bonus \\
$(Bonus \in \mathbb{R}_+)$ \\
$Big =$ valore molto grande \\
$(Big \in \mathbb{R}_+)$

\subsubsection{Variabili}
$C_{i,j}:i \in I, j \in J =\# \text{ condensatori del tipo i e di categoria j prodotti}$ \\
$(C_{i,j} \in \mathbb{Z}_+)$ \\
$M_{m,a}:m \in M, a \in A =\# \text{ unità di materiale m acquistato dall'azienda a}$ \\
$(M_{m,a} \in \mathbb{Z}_+)$ \\
$T_{i} =$ vera se vengono prodotti condensatori a $1000 \mu F$ di tipo i \\
$(T_{i} \in \{ 0,1\} )$ \\
$U_{a}:a \in A =$ vera se vengono acquistato dell'alluminio dall'azienda a \\
$(U_{a} \in \{ 0,1\} )$ \\
$U_{ex} =$ vera se viene acquistato alluminio da almeno un azienda diversa da Azienda 3 \\
$(U_{ex} \in \{ 0,1\} )$ \\
$P =$ vera se il numero di unità di materiale acquistato da Azienda 3 è maggiore o uguale a $L$ \\
$(P \in \{ 0,1\} )$ \\
$S =U_{"Azienda 3"} \land P$, vera se è possibile ottenere il premio di Azienda 3 \\
$(S \in \{ 0,1\} )$

\subsubsection{Funzione Obiettivo e Vincoli}
\begin{gather*}
\text{max } \underbrace{\sum_{i \in I,j \in J}^{} V_{i,j}*C_{i,j}}_{ricavo} - \underbrace{\sum_{m \in M,a \in A} A_{m,a}*M_{m,a}}_{costi} + \underbrace{Bonus*S}_{premio} \\
\text{s.t. } \sum_{i \in I,j \in J} E_{m,i,j}*C_{i,j} \leq \sum_{a \in A} M_{m,a} :\forall m \in M \\
\sum_{a \in A} M_{m,a} \leq X_{m} :\forall m \in M \\
\sum_{i \in I} C_{i,j} \geq M_{j} :\forall j \in J \\
\sum_{j \in J} C_{"Elettrolitico",j} \geq \frac{Per}{100}*\sum_{i \in I,j \in J} C_{i,j} \\
\sum_{m \in M} M_{m,"Azienda 1"} \geq K \\
C_{i,"1000 \mu F"} \leq Big*T_{i} :\forall i \in I \\
\sum_{i \in I} T_{i} = 1 \\
M_{"Alluminio",a} \leq Big*U_{a} :\forall a \in A \\
\sum_{a \in A-\{"Azienda 3"\}} U_{a} \leq card(A-\{"Azienda 3"\})*U_{ex} \\
U_{ex} + U_{"Azienda 3"} = 1 \\
\sum_{m \in M} M_{m,"Azienda 3"} -L\leq Big*P \\
U_{"Azienda 3"} + P \geq 2*S
\end{gather*}
\section{Note su AMPL}
Per il calcolo della soluzione ottima è stato utilizzato il solver cplex.\\
Elenco dei File:
\begin{itemize}
\item{File del Modello: CondensSrl.mod}
\item{File dei Dati: CondensSrl.dat}
\item{File dei Comandi: CondensSrl.run}
\end{itemize}
I nomi delle variabili e dei parametri differiscono da quelli presenti in questa relazione per esigenze di spazio.
Tuttavia essendo sia il file del modello che quello dei dati adeguatamente commentati, il codice AMPL risulta sufficientemente espressivo.
\subsection{Soluzione ottima al problema}
Valore ottimo della funzione obiettivo o guadagno massimo: 6812\\
Valori ottimi di $C_{i,j}$ o numero di condensatori venduti: \\
\begin{center}
\begin{tabular}{|c|c|c|}
\hline
             & Elettrolitico & Ceramico \\
\hline
$500 \mu F$  &       0       &    50    \\
\hline
$1000 \mu F$ &       212     &    0     \\
\hline
$1500 \mu F$ &       0       &    170   \\
\hline
\end{tabular}
\end{center}
Valori ottimi di $M_{m,a}$ o numero di materiali acquistati: \\
\begin{center}
\begin{tabular}{|c|c|c|c|}
\hline
         & Alluminio & Ceramica & Rame \\
\hline
Azienda1 &     0     &    100   &  0   \\
\hline
Azienda2 &     0     &    0     &  2434\\
\hline
Azienda3 &     2434  &    850   &  0   \\
\hline
\end{tabular}
\end{center}
\subsubsection{Valori delle variabili booleane}
\begin{itemize}
\item{$T_{"Elettrolitico"}=1$, $T_{"Ceramico"}=0$}
\item{$U_{"Azienda 1"}=0$, $U_{"Azienda 2"}=0$, $U_{"Azienda 3"}=1$, $U_{ex}=0$}
\item{$P=1$, $S=1$}
\end{itemize}
\subsection{Considerazioni sulla soluzione ottima}
Come previsto nessuna variabile booleana ha assunto valori spuri in quanto condurrebbero a soluzioni ammissibili ma non ottime. \\
Nonostante Azienda 3 venda alluminio ad un prezzo superiore rispetto alla concorrenza, il contratto di esclusività risulta vantaggioso e quindi, pur di ottenere il premio, si è disposti ad avere dei costi maggiori.
\end{document}